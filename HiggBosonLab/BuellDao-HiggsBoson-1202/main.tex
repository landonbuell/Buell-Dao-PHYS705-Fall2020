% ================
% Landon Buell
% Elena Long
% PHYS 705.01 - Lab00
% 21 Sept 2020
% ================

\RequirePackage{lineno}

\documentclass[aps,prl,twocolumn,superscriptaddress,groupedaddress,nofootinbib]{revtex4-2}  % for review and submission
%\documentclass[aps,preprint,superscriptaddress,groupedaddress]{revtex4-2}  % for double-spaced preprint

\usepackage{graphicx}  % needed for figures
\usepackage{dcolumn}   % needed for some tables
\usepackage{bm}                 % for math
\usepackage{amssymb,amsmath}    % for math
\usepackage{verbatim}
\usepackage{color}
\usepackage[english]{babel}
\usepackage{hyperref}

% ================================

\begin{document}

\title{Rediscoerving the Higgs Boson}
\author{Landon Buell and Tan Dao} 
\email[Corresponding authors: ]{lhb1007@wildcats.unh.edu, tni24@wildcats.unh.edu}
\affiliation{University of New Hampshire, Durham, NH, 03824, USA} 
\date{\today}

\begin{abstract}

\textcolor{red}{Need to write Abstract here}

\end{abstract}

\maketitle 

% ================================================================

\paragraph{}\emph{Introduction-} The standard model is the accepted theory that governs humanity's understanding of sub-atomic particle physics. Its is composed of multiple kinds of quarks, leptons, and bosons, each of which represent the universe at it's most fundamental and indivisible level \cite{Oerter}. Collectively, the particles in the standard model describe the strong nuclear force, weak nuclear force, electromagnetic force, as well as how the particles all interact with each other. While is consistent within it's own theory, has shown experimental validity and incredible predictive prowess, the standard model is still an incomplete theory of particle physics \cite{Englert,Mann}.

\paragraph{}

\paragraph{}Modern theories about the universe suggest that in the time immediately after the big-bang, fundamental particles were formed, but did not interact gravitationally - as if they had no mass \cite{Englert}

\hspace{1cm}

% ================================================================

\paragraph{}\emph{Methods-} 

\hspace{1cm}
\paragraph{}\emph{Discussions-} 


\hspace{1cm}
\paragraph*{}\emph{Conclusions-}



% ================================================================

\vspace{4cm}
\bibliographystyle{apsrev4-2}
\bibliography{bibliography}

\begin{thebibliography}{9}

\bibitem{Englert}
F. Englert and R. Brout, “Broken symmetry and the mass of gauge vector mesons”, Phys.Rev. Lett. 13 (1964) 321–323, doi:10.1103/PhysRevLett.13.321.

\bibitem{Higgs}
P.W. Higgs, “Broken symmetries and the masses of gauge bosons”, Phys. Rev. Lett. 13 (1964) 508–509, doi:10.1103/PhysRevLett.13.508.

\bibitem{Mann}
Mann, Robert. \textit{An Introduction to Particle Physics and the Standard Model}. Taylor \& Francis, 2010. 

\bibitem{Oerter}
Oerter, Robert. \textit{The Theory of Almost Everything: The Standard Model, the Unsung Triumph of Modern Physics}. Pi Press, 2006. 

\bibitem{ROOT}
Rene Brun and Fons Rademakers, ROOT - An Object Oriented Data Analysis Framework, Proceedings AIHENP'96 Workshop, Lausanne, Sep. 1996, Nucl. Inst. & Meth. in Phys. Res. A 389 (1997) 81-86.


\end{thebibliography}



% ================================

\end{document}
%
% ****** End of file template.aps ******
























